\documentclass{article}
\usepackage{nopageno}
\usepackage[utf8]{inputenc}
\usepackage[a4paper, total={6in, 9in}]{geometry}

\title{Market Trends, November 2022 Newsletter}
\author{}
\date{}
\begin{document}

\maketitle

\section*{\normalsize Vacancy Rates Rose by All Measures}

RealPage’s national vacancy rate for investment-grade apartments rose 90 bps in the third quarter of 2022 - and 120 bps from the third quarter of 2021 – to 4.1 \%. This increase in vacancies spanned all regions but was most pronounced in the South, where rates increased more than one percentage point to 4.9\%. Vacancy rates rose 80 bps in both the Midwest and West as well as 70 bps in the Northeast.

\medskip

\noindent The U.S. Census Bureau’s rental vacancy rate for all apartments (in buildings with five or more units) rose 30 bps to 6.8\%, on par with the vacancy rate in the third quarter of 2021.  

\section*{\normalsize (See PPT) Figure 1. US Rental Apartment Vacancy Rate (5+ Units)}

\begin{center}
    \begin{tabular}{ |p{3cm}||p{1.5cm}|p{1.5cm}|p{1.5cm}|p{1.5cm}|p{1.5cm}|  }
    \hline
     Multifamily Vacant (\%), RealPage & \centering 3Q 2022 & \centering 2Q 2022 & \centering Change Last Qtr & \centering 3Q 2021 & Change Year Ago\\
     \hline
     Northeast & 2.9 & 2.2 & 0.7 & 2.4 & 0.5\\
     \hline
     Midwest & 3.6 & 2.8 & 0.8 & 2.9 & 0.7\\
     \hline
     South & 4.9 & 3.1 & 0.8 & 2.5 & 1.4\\
     \hline
     West & 3.9 & 3.1 & 0.8 & 2.5 & 1.4\\
     \hline
     \textbf{Total U.S.} & 4.1 & 3.2 & 0.9 & 2.9 & 1.2\\
     \hline
    \end{tabular}
\end{center}

\section*{\normalsize Multifamily Permits and Completions Rose, Starts Fell Slightly}

Multifamily permits (5+ units in structure) rose 0.9\% from last quarter to a seasonally adjusted annual rate (SAAR) of 647,000, a 12.2\% increase from 3Q 2021. On a not-seasonally adjusted basis, permit levels (2+ units in structure) rose 5.3\% in the South but fell 16.4\% in the Northeast, 2.6\% in the Midwest, and 4.0\% in the West. Compared to 2Q 2021, 2+ permit levels were up across all regions except for the Northeast.

\medskip

\noindent Starts (5+) declined 1.8\% this quarter to a SAAR of 534,000, a 17.2\% increase from the previous year. Despite the modest decline from last quarter, this represents the highest level of third quarter starts (5+) dating back to Q3 1985.

\medskip

\noindent Multifamily completions rose a modest 1.4\% from last quarter $–$ and 5.2\% from last year $–$ to a SAAR of 365,700. While completions have increased for two consecutive quarters, they still lag significantly behind starts, predominantly due to continued difficulties in staffing and supply chain disruptions.  

\begin{center}
    \begin{tabular}{ |p{3cm}||p{1.5cm}|p{1.5cm}|p{1.5cm}|p{1.5cm}|p{1.5cm}|  }
    \hline
     Multifamily Vacant (\%), RealPage & \centering 3Q 2022 & \centering 2Q 2022 & \centering Change Last Qtr & \centering 3Q 2021 & Change Year Ago\\
     \hline
     Northeast & 18.8 & 22.5 & -3.7 & 19.9 & -1.1\\
     \hline
     Midwest & 25.9 & 26.6 & -0.7 & 24.1 & 1.8\\
     \hline
     South & 85.7 & 81.4 & 4.3 & 70.8 & 14.9\\
     \hline
     West & 45.1 & 47.0 & -1.9 & 44.2 & 0.9\\
     \hline
     \textbf{Total U.S.} & 175.6 & 177.7 & -2.1 & 159.1 & 16.5\\
     \hline
    \end{tabular}
    \begin{tablenotes}
        \small
        \item \textbf{(See PPT) Figure 2. U.S. Multifamily Permit Issuance, Starts and Completions (in 000s)}
    \end{tablenotes}
\end{center}

\section*{\normalsize Multifamily Absorption Negative for Second Straight Quarter}

Net absorptions of investment-grade, market-rate apartments tracked by RealPage fell for the second consecutive quarter to -82,095 in 3Q 2022 – down from -60,404 in 2Q 2022 $–$ and marking the third lowest absorption level since RealPage started tracking this data in 2000. The trailing four-quarter sum fell 81.8\% to just 77,936, down 87.2\% from the previous year. This marks the lowest four-quarter sum since 4Q 2009. 

\bigskip

\section*{\normalsize (See PPT) Figure 3. Net Absorptions (Investment-Grade, Market Rate Apartments, 000s)}

\bigskip

\section*{\normalsize Rent Growth Continues to Moderate from Record Highs}

Same-store apartment rents for professionally managed apartments tracked by RealPage rose 10.5\% year-over-year in 3Q 2022, marking a deceleration from the 14.5\% growth recorded last quarter and the 15.3\% growth in 1Q 2022. 
The South experienced the largest annual rent growth for the sixth consecutive quarter at 11.6\%, while rents grew 10.5\% in the West, 9.7\% in the Northeast and 8.9\% in the Midwest. 

\medskip

\noindent The CPI rent index, which covers all rental housing, rose 5.8\% year-over-year, up 50 basis points from last quarter. 

\bigskip

\section*{\normalsize (See PPT) Figure 4: U.S. Rent Inflation, Annual Rate}

\bigskip

\section*{\normalsize Apartment Transaction Volumes Decline}

In the apartment transaction market tracked by Real Capital Analytics, sales volume decreased 21.3\% from 2Q 2022 to \$74.1 billion, down 17.2\% from the previous year. 

\medskip

\noindent The market value of investment-grade apartments, as measured by the National Council of Real Estate Investment Fiduciaries (NCREIF), rose a modest 0.3\% from last quarter and 14.0\% from a year ago. 

\medskip

\noindent Cap rates remained at record low levels of 4.6\% in 3Q 2022, unchanged from last quarter. The average price per unit also fell by 2.2\% from 2Q 2022 to \$240,829, according to data from Real Capital Analytics. 

\bigskip

\section*{\normalsize (See PPT) Figure 5. Apartment Transaction Volume (\$Billions)}

\bigskip

\section*{\normalsize GSEs Still Back Largest Share of Multifamily Debt}

Total multifamily mortgage debt outstanding increased 2.1\% to \$1.96 trillion in Q2 2022, up 8.5\% from Q2 2021. Fannie and Freddie accounted for 39.7\% (\$779 billion) of the nation’s \$1.96 trillion in total multifamily mortgage debt outstanding in 2Q 2022, down 50 bps from 1Q 2022. Ginnie Mae held an additional 7.1\%, meaning that 46.8\% of all multifamily mortgage debt outstanding is backed by the federal government, down from its peak of 48.3\% recorded in 1Q 2021.

\medskip

\noindent The share of multifamily debt held by depository institutions rose 90 bps from the first quarter – and 1.4 percentage points since 1Q 2021 – to 31\% in 2Q 2022. Depositories are the second-largest holders of multifamily debt after the GSEs.

\medskip

\noindent Mortgage-backed securities (MBS) accounted for 27.0\% of the total outstanding debt at \$530 million, while commercial mortgage-backed securities (CMBS) accounted for 3.5\% of total outstanding debt, down 20 bps from the previous quarter. 
Life companies held 9.6\% of multifamily mortgage debt in 2Q 2022, while other independent institutions accounted for the remaining 9.1\%.  

\bigskip

\section*{\normalsize (See PPT) Figure 6. Share of Multifamily Mortgage Debt Outstanding by Fiscal Quarter}

\bigskip

\hline

\smallskip

\small{Technical Note. Historical statistics from RealPage, Inc., reflect consolidation of records from the firm’s MPF Research and Axiometrics data sets into a single information source. While individual performance metrics such as monthly rent and net absorption shift to some degree, trends follow patterns reported previously.}



\end{document}
